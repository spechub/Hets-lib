
\chapter{Conclusion and Outlook}
\label{cha:outlook}


Ideas:

Note that Hilbert calculi are not well suited for automatic proving

Syntactical/notational conventions are easily introduced on paper, but in
Isabelles framework this is not so straightforward. Example: rule (new-distinct)
and a failed proof attempt of a slight modification with the center program $p$
missing.

rules new-*: are not quite suitable\ldots it would be much better to use the
conjunctive (if i were) form, but this is not possible, since quantification is
only possible at the level of boolean values and not of dsef programs. (see State).

the subtype D is not a real subtype; this makes proofs blurred by notation and
for complex formulas in D expressed by sequences of programs in T constitutes a
real problem.


Isabelle's arithmetical reasoner is not quite as strong as one would
hope. Therefore, a proof involving typical advanced arithmetic (e.g. for
numerical algorithms on while-loops) might become cumbersome.

%%% Local Variables: 
%%% mode: latex
%%% TeX-master: "main"
%%% End: 
