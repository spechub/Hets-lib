%
\begin{isabellebody}%
\def\isabellecontext{Parsec}%
%
\isamarkupheader{A Deterministic Parser Monad with Fall Back Alternatives%
}
\isamarkuptrue%
\isacommand{theory}\ Parsec\ {\isacharequal}\ PDL\ {\isacharplus}\ MonEq{\isacharcolon}\isamarkupfalse%
%
\label{sec:parsec-thy}
%
\begin{isamarkuptext}%
In a typical implementation of this parser monad, \isa{T} would have the 
  form \isa{T\ A\ {\isacharequal}\ {\isacharparenleft}S\ {\isasymRightarrow}\ {\isacharparenleft}E\ {\isacharplus}\ A{\isacharparenright}\ {\isasymtimes}\ S{\isacharparenright}}, i.e. it would be a state monad (over states
  $S$) with exceptions of type $E$. The fall back alternative \isa{q} in
  \isa{p{\isasymparallel}q} would then only be used if \isa{p} failed to terminate.
  \label{isa:parsec-spec}%
\end{isamarkuptext}%
\isamarkuptrue%
\isacommand{consts}\isanewline
\ \ item\ \ \ \ \ \ \ {\isacharcolon}{\isacharcolon}\ {\isachardoublequote}nat\ T{\isachardoublequote}\ \ \ \ \ \ %
\isamarkupcmt{Parses exactly one character (natural number)%
}
\isanewline
\ \ fail\ \ \ \ \ \ \ {\isacharcolon}{\isacharcolon}\ {\isachardoublequote}{\isacharprime}a\ T{\isachardoublequote}\ \ \ \ \ \ \ %
\isamarkupcmt{Always fails%
}
\isanewline
\ \ alt\ \ \ \ \ \ \ \ {\isacharcolon}{\isacharcolon}\ {\isachardoublequote}{\isacharprime}a\ T\ {\isasymRightarrow}\ {\isacharprime}a\ T\ {\isasymRightarrow}\ {\isacharprime}a\ T{\isachardoublequote}\ {\isacharparenleft}\isakeyword{infixl}\ {\isachardoublequote}{\isasymparallel}{\isachardoublequote}\ {\isadigit{1}}{\isadigit{4}}{\isadigit{0}}{\isacharparenright}\ %
\isamarkupcmt{Prefer first parser, but fall back on second if necessary%
}
\isanewline
\ \ getInput\ \ \ {\isacharcolon}{\isacharcolon}\ {\isachardoublequote}nat\ list\ T{\isachardoublequote}\ %
\isamarkupcmt{read the current state%
}
\isanewline
\ \ setInput\ \ \ {\isacharcolon}{\isacharcolon}\ {\isachardoublequote}nat\ list\ {\isasymRightarrow}\ unit\ T{\isachardoublequote}\ \isanewline
\isanewline
\isanewline
\isamarkupfalse%
\isacommand{constdefs}\ \isanewline
\ \ eot\ {\isacharcolon}{\isacharcolon}\ {\isachardoublequote}bool\ T{\isachardoublequote}\isanewline
\ \ {\isachardoublequote}eot\ {\isasymequiv}\ {\isacharparenleft}do\ {\isacharbraceleft}i\ {\isasymleftarrow}\ getInput{\isacharsemicolon}\ ret\ {\isacharparenleft}null\ i{\isacharparenright}{\isacharbraceright}{\isacharparenright}{\isachardoublequote}\isanewline
\ \ Eot\ {\isacharcolon}{\isacharcolon}\ {\isachardoublequote}bool\ D{\isachardoublequote}\isanewline
\ \ {\isachardoublequote}Eot\ {\isasymequiv}\ {\isasymUp}\ eot{\isachardoublequote}\isanewline
\ \ GetInput\ {\isacharcolon}{\isacharcolon}\ {\isachardoublequote}nat\ list\ D{\isachardoublequote}\isanewline
\ \ {\isachardoublequote}GetInput\ {\isasymequiv}\ {\isasymUp}\ getInput{\isachardoublequote}\isamarkupfalse%
%
\begin{isamarkuptext}%
\isa{GetInput} and \isa{Eot} are the abstractions in \isa{{\isacharprime}a\ D} of the
resp. lower case terms in \isa{{\isacharprime}a\ T}.%
\end{isamarkuptext}%
\isamarkuptrue%
\isacommand{axioms}\isanewline
\ \ dsef{\isacharunderscore}getInput{\isacharcolon}\ {\isachardoublequote}dsef\ getInput{\isachardoublequote}\isanewline
\ \ fail{\isacharunderscore}bot{\isacharcolon}\ {\isachardoublequote}{\isasymturnstile}\ {\isacharbrackleft}{\isacharhash}\ fail{\isacharbrackright}{\isacharparenleft}{\isasymlambda}x{\isachardot}\ Ret\ False{\isacharparenright}{\isachardoublequote}\isanewline
\ \ eot{\isacharunderscore}item{\isacharcolon}\ {\isachardoublequote}{\isasymturnstile}\ Eot\ {\isasymlongrightarrow}\isactrlsub D\ {\isacharbrackleft}{\isacharhash}\ x{\isasymleftarrow}item{\isacharbrackright}{\isacharparenleft}Ret\ False{\isacharparenright}{\isachardoublequote}\isanewline
\ \ set{\isacharunderscore}get{\isacharcolon}\ \ {\isachardoublequote}{\isasymturnstile}\ {\isasymlangle}setInput\ x{\isasymrangle}{\isacharparenleft}{\isasymlambda}u{\isachardot}\ GetInput\ {\isacharequal}\isactrlsub D\ Ret\ x{\isacharparenright}{\isachardoublequote}\isanewline
\ \ get{\isacharunderscore}item{\isacharcolon}\ {\isachardoublequote}{\isasymturnstile}\ GetInput\ {\isacharequal}\isactrlsub D\ Ret\ {\isacharparenleft}y{\isacharhash}ys{\isacharparenright}\ {\isasymlongrightarrow}\isactrlsub D\ {\isasymlangle}x{\isasymleftarrow}item{\isasymrangle}{\isacharparenleft}Ret\ {\isacharparenleft}x\ {\isacharequal}\ y{\isacharparenright}\ {\isasymand}\isactrlsub D\ GetInput\ {\isacharequal}\isactrlsub D\ Ret\ ys{\isacharparenright}{\isachardoublequote}\isanewline
\ \ altB{\isacharunderscore}iff{\isacharcolon}\ {\isachardoublequote}{\isasymturnstile}\ {\isacharbrackleft}{\isacharhash}\ x{\isasymleftarrow}p{\isasymparallel}q{\isacharbrackright}{\isacharparenleft}P\ x{\isacharparenright}\ {\isasymlongleftrightarrow}\isactrlsub D\ {\isacharparenleft}\ {\isacharbrackleft}{\isacharhash}\ x{\isasymleftarrow}p{\isacharbrackright}{\isacharparenleft}P\ x{\isacharparenright}\ {\isasymand}\isactrlsub D\ {\isasymlangle}x{\isasymleftarrow}p{\isasymrangle}{\isacharparenleft}Ret\ True{\isacharparenright}\ {\isacharparenright}\ {\isasymor}\isactrlsub D\ \isanewline
\ \ \ \ \ \ \ \ \ \ \ \ \ \ \ \ \ \ \ \ \ \ \ \ \ \ \ \ \ \ \ \ \ \ \ \ \ {\isacharparenleft}\ {\isacharbrackleft}{\isacharhash}\ x{\isasymleftarrow}q{\isacharbrackright}{\isacharparenleft}P\ x{\isacharparenright}\ {\isasymand}\isactrlsub D\ {\isacharbrackleft}{\isacharhash}\ x{\isasymleftarrow}p{\isacharbrackright}{\isacharparenleft}Ret\ False{\isacharparenright}\ {\isacharparenright}{\isachardoublequote}\isanewline
\ \ altD{\isacharunderscore}iff{\isacharcolon}\ {\isachardoublequote}{\isasymturnstile}\ {\isasymlangle}x{\isasymleftarrow}p{\isasymparallel}q{\isasymrangle}{\isacharparenleft}P\ x{\isacharparenright}\ {\isasymlongleftrightarrow}\isactrlsub D\ {\isasymlangle}x{\isasymleftarrow}p{\isasymrangle}{\isacharparenleft}P\ x{\isacharparenright}\ {\isasymor}\isactrlsub D\ {\isacharparenleft}{\isasymlangle}x{\isasymleftarrow}q{\isasymrangle}{\isacharparenleft}P\ x{\isacharparenright}\ {\isasymand}\isactrlsub D\ {\isacharbrackleft}{\isacharhash}\ x{\isasymleftarrow}p{\isacharbrackright}{\isacharparenleft}Ret\ False{\isacharparenright}{\isacharparenright}{\isachardoublequote}\isanewline
\ \ determ{\isacharcolon}\ \ \ {\isachardoublequote}{\isasymturnstile}\ {\isasymlangle}x{\isasymleftarrow}p{\isasymrangle}{\isacharparenleft}P\ x{\isacharparenright}\ {\isasymlongleftrightarrow}\isactrlsub D\ {\isacharbrackleft}{\isacharhash}\ x{\isasymleftarrow}p{\isacharbrackright}{\isacharparenleft}P\ x{\isacharparenright}\ {\isasymand}\isactrlsub D\ {\isasymlangle}x{\isasymleftarrow}p{\isasymrangle}{\isacharparenleft}Ret\ True{\isacharparenright}{\isachardoublequote}\isamarkupfalse%
%
\begin{isamarkuptext}%
Axiom \isa{{\isachardoublequote}determ{\isachardoublequote}} is the typical relationship between \isa{{\isasymlangle}p{\isasymrangle}P} and \isa{{\isacharbrackleft}{\isacharhash}\ p{\isacharbrackright}P} 
  when no non-determinism is involved. Axioms \isa{{\isachardoublequote}altB{\isacharunderscore}iff{\isachardoublequote}\ {\isachardoublequote}altD{\isacharunderscore}iff{\isachardoublequote}} describe the 
  fall back behaviour of the alternative operation.%
\end{isamarkuptext}%
\isamarkuptrue%
%
\begin{isamarkuptext}%
\isa{dsef\ getInput} implies \isa{dsef\ eot}.%
\end{isamarkuptext}%
\isamarkuptrue%
\isacommand{lemma}\ dsef{\isacharunderscore}eot{\isacharcolon}\ {\isachardoublequote}dsef\ eot{\isachardoublequote}\isanewline
\ \ \isamarkupfalse%
\isacommand{by}\ {\isacharparenleft}simp\ add{\isacharcolon}\ eot{\isacharunderscore}def\ dsef{\isacharunderscore}seq\ dsef{\isacharunderscore}ret\ dsef{\isacharunderscore}getInput{\isacharparenright}\isamarkupfalse%
%
\begin{isamarkuptext}%
Another way to state the properties of alternation (for the diamond operator).%
\end{isamarkuptext}%
\isamarkuptrue%
\isacommand{axioms}\isanewline
altD{\isacharunderscore}left{\isacharcolon}\ {\isachardoublequote}{\isasymturnstile}\ {\isasymlangle}p{\isasymrangle}P\ {\isasymlongrightarrow}\isactrlsub D\ {\isasymlangle}p{\isasymparallel}q{\isasymrangle}P{\isachardoublequote}\isanewline
altD{\isacharunderscore}right{\isacharcolon}\ {\isachardoublequote}{\isasymturnstile}\ {\isasymlangle}q{\isasymrangle}P\ {\isasymlongrightarrow}\isactrlsub D\ {\isasymlangle}p{\isasymrangle}{\isacharparenleft}{\isasymlambda}x{\isachardot}\ Ret\ True{\isacharparenright}\ {\isasymor}\isactrlsub D\ {\isasymlangle}p{\isasymparallel}q{\isasymrangle}P{\isachardoublequote}\isamarkupfalse%
%
\begin{isamarkuptext}%
Proof that \isa{Eot} actually is just an abbreviation.%
\end{isamarkuptext}%
\isamarkuptrue%
\isacommand{lemma}\ Eot{\isacharunderscore}GetInput{\isacharcolon}\ {\isachardoublequote}Eot\ {\isacharequal}\ {\isacharparenleft}GetInput\ {\isacharequal}\isactrlsub D\ Ret\ {\isacharbrackleft}{\isacharbrackright}{\isacharparenright}{\isachardoublequote}\isanewline
\isamarkupfalse%
\isacommand{proof}\ {\isacharminus}\isanewline
\ \ \isamarkupfalse%
\isacommand{have}\ null{\isacharunderscore}eq{\isacharunderscore}nil{\isacharcolon}\ {\isachardoublequote}\ {\isasymforall}x{\isachardot}\ null\ x\ {\isacharequal}\ {\isacharparenleft}x\ {\isacharequal}\ {\isacharbrackleft}{\isacharbrackright}{\isacharparenright}{\isachardoublequote}\isanewline
\ \ \isamarkupfalse%
\isacommand{proof}\isanewline
\ \ \ \ \isamarkupfalse%
\isacommand{fix}\ x\ \isamarkupfalse%
\isacommand{show}\ {\isachardoublequote}null\ x\ {\isacharequal}\ {\isacharparenleft}x\ {\isacharequal}\ {\isacharbrackleft}{\isacharbrackright}{\isacharparenright}{\isachardoublequote}\isanewline
\ \ \ \ \isamarkupfalse%
\isacommand{proof}\ {\isacharparenleft}cases\ x{\isacharparenright}\ \isanewline
\ \ \ \ \ \ \isamarkupfalse%
\isacommand{assume}\ {\isachardoublequote}x\ {\isacharequal}\ {\isacharbrackleft}{\isacharbrackright}{\isachardoublequote}\ \isamarkupfalse%
\isacommand{thus}\ {\isachardoublequote}null\ x\ {\isacharequal}\ {\isacharparenleft}x\ {\isacharequal}\ {\isacharbrackleft}{\isacharbrackright}{\isacharparenright}{\isachardoublequote}\ \isamarkupfalse%
\isacommand{by}\ simp\isanewline
\ \ \ \ \isamarkupfalse%
\isacommand{next}\isanewline
\ \ \ \ \ \ \isamarkupfalse%
\isacommand{fix}\ a\ list\ \isamarkupfalse%
\isacommand{assume}\ {\isachardoublequote}x\ {\isacharequal}\ {\isacharparenleft}a{\isacharhash}list{\isacharparenright}{\isachardoublequote}\ \isamarkupfalse%
\isacommand{thus}\ {\isachardoublequote}null\ x\ {\isacharequal}\ {\isacharparenleft}x\ {\isacharequal}\ {\isacharbrackleft}{\isacharbrackright}{\isacharparenright}{\isachardoublequote}\ \isamarkupfalse%
\isacommand{by}\ simp\isanewline
\ \ \ \ \isamarkupfalse%
\isacommand{qed}\isanewline
\ \ \isamarkupfalse%
\isacommand{qed}\isanewline
\ \ \isamarkupfalse%
\isacommand{show}\ {\isacharquery}thesis\isanewline
\ \ \isamarkupfalse%
\isacommand{by}{\isacharparenleft}simp\ add{\isacharcolon}\ Eot{\isacharunderscore}def\ eot{\isacharunderscore}def\ GetInput{\isacharunderscore}def\ MonEq{\isacharunderscore}def\ liftM{\isadigit{2}}{\isacharunderscore}def\ \isanewline
\ \ \ \ \ \ \ \ \ \ \ \ \ \ \ \ \ \ dsef{\isacharunderscore}getInput\ Abs{\isacharunderscore}Dsef{\isacharunderscore}inverse\ Dsef{\isacharunderscore}def\ Ret{\isacharunderscore}def\ null{\isacharunderscore}eq{\isacharunderscore}nil{\isacharparenright}\isanewline
\isamarkupfalse%
\isacommand{qed}\isanewline
\isanewline
\isamarkupfalse%
\isacommand{lemma}\ GetInput{\isacharunderscore}item{\isacharunderscore}fail{\isacharcolon}\ {\isachardoublequote}{\isasymturnstile}\ GetInput\ {\isacharequal}\isactrlsub D\ Ret\ {\isacharbrackleft}{\isacharbrackright}\ {\isasymlongrightarrow}\isactrlsub D\ {\isacharbrackleft}{\isacharhash}\ item{\isacharbrackright}{\isacharparenleft}{\isasymlambda}x{\isachardot}\ Ret\ False{\isacharparenright}{\isachardoublequote}\isanewline
\ \ \isamarkupfalse%
\isacommand{apply}{\isacharparenleft}rule\ subst{\isacharbrackleft}OF\ Eot{\isacharunderscore}GetInput{\isacharbrackright}{\isacharparenright}\isanewline
\ \ \isamarkupfalse%
\isacommand{by}\ {\isacharparenleft}rule\ eot{\isacharunderscore}item{\isacharparenright}\isamarkupfalse%
%
\begin{isamarkuptext}%
We can show that an alternative parser terminates iff one of its constituent
  parsers does.%
\end{isamarkuptext}%
\isamarkuptrue%
\isacommand{lemma}\ par{\isacharunderscore}term{\isacharcolon}\ {\isachardoublequote}{\isasymturnstile}\ {\isasymlangle}x\ {\isasymleftarrow}\ p{\isasymparallel}q{\isasymrangle}{\isacharparenleft}Ret\ True{\isacharparenright}\ {\isasymlongleftrightarrow}\isactrlsub D\ {\isasymlangle}x{\isasymleftarrow}p{\isasymrangle}{\isacharparenleft}Ret\ True{\isacharparenright}\ {\isasymor}\isactrlsub D\ {\isasymlangle}x{\isasymleftarrow}q{\isasymrangle}{\isacharparenleft}Ret\ True{\isacharparenright}{\isachardoublequote}\isanewline
\isamarkupfalse%
\isacommand{proof}\ {\isacharparenleft}rule\ pdl{\isacharunderscore}iffI{\isacharparenright}\isanewline
\ \ \isamarkupfalse%
\isacommand{have}\ {\isachardoublequote}{\isasymturnstile}\ {\isacharparenleft}\ {\isasymlangle}x{\isasymleftarrow}p{\isasymparallel}q{\isasymrangle}{\isacharparenleft}Ret\ True{\isacharparenright}\ {\isasymlongrightarrow}\isactrlsub D\ {\isasymlangle}x{\isasymleftarrow}p{\isasymrangle}{\isacharparenleft}Ret\ True{\isacharparenright}\ {\isasymor}\isactrlsub D\ {\isasymlangle}x{\isasymleftarrow}q{\isasymrangle}{\isacharparenleft}Ret\ True{\isacharparenright}\ {\isasymand}\isactrlsub D\ {\isacharbrackleft}{\isacharhash}\ x{\isasymleftarrow}p{\isacharbrackright}{\isacharparenleft}Ret\ False{\isacharparenright}\ {\isacharparenright}\ {\isasymlongrightarrow}\isactrlsub D\ \isanewline
\ \ \ \ \ \ \ \ \ \ {\isasymlangle}x{\isasymleftarrow}p{\isasymparallel}q{\isasymrangle}{\isacharparenleft}Ret\ True{\isacharparenright}\ {\isasymlongrightarrow}\isactrlsub D\ {\isasymlangle}x{\isasymleftarrow}p{\isasymrangle}{\isacharparenleft}Ret\ True{\isacharparenright}\ {\isasymor}\isactrlsub D\ {\isasymlangle}x{\isasymleftarrow}q{\isasymrangle}{\isacharparenleft}Ret\ True{\isacharparenright}{\isachardoublequote}\isanewline
\ \ \ \ \isamarkupfalse%
\isacommand{by}\ {\isacharparenleft}simp\ add{\isacharcolon}\ pdl{\isacharunderscore}taut{\isacharparenright}\isanewline
\ \ \isamarkupfalse%
\isacommand{moreover}\ \isamarkupfalse%
\isacommand{note}\ pdl{\isacharunderscore}iffD{\isadigit{1}}{\isacharbrackleft}OF\ altD{\isacharunderscore}iff{\isacharbrackright}\isanewline
\ \ \isamarkupfalse%
\isacommand{ultimately}\ \isamarkupfalse%
\isacommand{show}\ \ {\isachardoublequote}{\isasymturnstile}\ {\isasymlangle}p\ {\isasymparallel}\ q{\isasymrangle}{\isacharparenleft}{\isasymlambda}x{\isachardot}\ Ret\ True{\isacharparenright}\ {\isasymlongrightarrow}\isactrlsub D\ {\isasymlangle}p{\isasymrangle}{\isacharparenleft}{\isasymlambda}x{\isachardot}\ Ret\ True{\isacharparenright}\ {\isasymor}\isactrlsub D\ {\isasymlangle}q{\isasymrangle}{\isacharparenleft}{\isasymlambda}x{\isachardot}\ Ret\ True{\isacharparenright}{\isachardoublequote}\ \isanewline
\ \ \ \ \isamarkupfalse%
\isacommand{by}\ {\isacharparenleft}rule\ pdl{\isacharunderscore}mp{\isacharparenright}\isanewline
\isamarkupfalse%
\isacommand{next}\isanewline
\ \ \isamarkupfalse%
\isacommand{have}\ {\isachardoublequote}{\isasymturnstile}\ {\isacharparenleft}\ {\isasymlangle}x{\isasymleftarrow}p{\isasymrangle}{\isacharparenleft}Ret\ True{\isacharparenright}\ {\isasymor}\isactrlsub D\ {\isasymlangle}x{\isasymleftarrow}q{\isasymrangle}{\isacharparenleft}Ret\ True{\isacharparenright}\ {\isasymand}\isactrlsub D\ {\isacharbrackleft}{\isacharhash}\ x{\isasymleftarrow}p{\isacharbrackright}{\isacharparenleft}Ret\ False{\isacharparenright}\ {\isasymlongrightarrow}\isactrlsub D\ {\isasymlangle}x{\isasymleftarrow}\ p\ {\isasymparallel}\ q{\isasymrangle}{\isacharparenleft}Ret\ True{\isacharparenright}\ {\isacharparenright}\ {\isasymlongrightarrow}\isactrlsub D\ \isanewline
\ \ \ \ \ \ \ \ \ \ {\isacharparenleft}\ {\isacharbrackleft}{\isacharhash}\ x{\isasymleftarrow}p{\isacharbrackright}{\isacharparenleft}Ret\ False{\isacharparenright}\ {\isasymlongleftrightarrow}\isactrlsub D\ \ {\isasymnot}\isactrlsub D\ {\isasymlangle}x{\isasymleftarrow}p{\isasymrangle}{\isacharparenleft}{\isasymnot}\isactrlsub D\ Ret\ False{\isacharparenright}\ {\isacharparenright}\ {\isasymlongrightarrow}\isactrlsub D\ \isanewline
\ \ \ \ \ \ \ \ \ \ \ {\isasymlangle}x{\isasymleftarrow}p{\isasymrangle}{\isacharparenleft}Ret\ True{\isacharparenright}\ {\isasymor}\isactrlsub D\ {\isasymlangle}x{\isasymleftarrow}q{\isasymrangle}{\isacharparenleft}Ret\ True{\isacharparenright}\ {\isasymlongrightarrow}\isactrlsub D\ {\isasymlangle}x{\isasymleftarrow}\ p\ {\isasymparallel}\ q{\isasymrangle}{\isacharparenleft}Ret\ True{\isacharparenright}{\isachardoublequote}\isanewline
\ \ \ \ \isamarkupfalse%
\isacommand{by}\ {\isacharparenleft}simp\ add{\isacharcolon}\ pdl{\isacharunderscore}taut{\isacharparenright}\isanewline
\ \ \isamarkupfalse%
\isacommand{moreover}\ \isanewline
\ \ \isamarkupfalse%
\isacommand{note}\ pdl{\isacharunderscore}iffD{\isadigit{2}}{\isacharbrackleft}OF\ altD{\isacharunderscore}iff{\isacharbrackright}\isanewline
\ \ \isamarkupfalse%
\isacommand{moreover}\ \isanewline
\ \ \isamarkupfalse%
\isacommand{note}\ box{\isacharunderscore}dmd{\isacharunderscore}rel\isanewline
\ \ \isamarkupfalse%
\isacommand{ultimately}\isanewline
\ \ \isamarkupfalse%
\isacommand{show}\ {\isachardoublequote}{\isasymturnstile}\ {\isasymlangle}x{\isasymleftarrow}p{\isasymrangle}{\isacharparenleft}Ret\ True{\isacharparenright}\ {\isasymor}\isactrlsub D\ {\isasymlangle}x{\isasymleftarrow}q{\isasymrangle}{\isacharparenleft}Ret\ True{\isacharparenright}\ {\isasymlongrightarrow}\isactrlsub D\ {\isasymlangle}x{\isasymleftarrow}\ p\ {\isasymparallel}\ q{\isasymrangle}{\isacharparenleft}Ret\ True{\isacharparenright}{\isachardoublequote}\isanewline
\ \ \ \ \isamarkupfalse%
\isacommand{by}\ {\isacharparenleft}rule\ pdl{\isacharunderscore}mp{\isacharunderscore}{\isadigit{2}}x{\isacharparenright}\isanewline
\isamarkupfalse%
\isacommand{qed}\isamarkupfalse%
%
\begin{isamarkuptext}%
The following two lemmas are immediate from the axioms.%
\end{isamarkuptext}%
\isamarkuptrue%
\isacommand{lemma}\ parI{\isadigit{1}}{\isacharcolon}\ {\isachardoublequote}{\isasymturnstile}\ \ {\isacharbrackleft}{\isacharhash}\ x{\isasymleftarrow}p{\isacharbrackright}{\isacharparenleft}P\ x{\isacharparenright}\ {\isasymand}\isactrlsub D\ {\isasymlangle}x{\isasymleftarrow}p{\isasymrangle}{\isacharparenleft}Ret\ True{\isacharparenright}\ {\isasymlongrightarrow}\isactrlsub D\ {\isacharbrackleft}{\isacharhash}\ x{\isasymleftarrow}p{\isasymparallel}q{\isacharbrackright}{\isacharparenleft}P\ x{\isacharparenright}{\isachardoublequote}\isamarkupfalse%
\isamarkupfalse%
\isamarkupfalse%
\isamarkupfalse%
\isamarkupfalse%
\isamarkupfalse%
\isamarkupfalse%
\isamarkupfalse%
\isamarkupfalse%
\isamarkupfalse%
\isanewline
\isanewline
\isamarkupfalse%
\isacommand{lemma}\ parI{\isadigit{2}}{\isacharcolon}\ {\isachardoublequote}{\isasymturnstile}\ {\isacharbrackleft}{\isacharhash}\ x{\isasymleftarrow}p{\isacharbrackright}{\isacharparenleft}Ret\ False{\isacharparenright}\ {\isasymand}\isactrlsub D\ {\isacharbrackleft}{\isacharhash}\ x{\isasymleftarrow}q{\isacharbrackright}{\isacharparenleft}P\ x{\isacharparenright}\ {\isasymlongrightarrow}\isactrlsub D\ {\isacharbrackleft}{\isacharhash}\ x{\isasymleftarrow}\ p{\isasymparallel}q{\isacharbrackright}{\isacharparenleft}P\ x{\isacharparenright}{\isachardoublequote}\isamarkupfalse%
\isamarkupfalse%
\isamarkupfalse%
\isamarkupfalse%
\isamarkupfalse%
\isamarkupfalse%
\isamarkupfalse%
\isamarkupfalse%
\isamarkupfalse%
\isamarkupfalse%
\isamarkupfalse%
%
\isamarkupsubsection{Specifying Simple Parsers in Terms of the Basic Ones%
}
\isamarkuptrue%
%
\label{isa:defined-parsers}
\isacommand{constdefs}\isanewline
\ \ sat\ \ \ \ \ \ \ \ {\isacharcolon}{\isacharcolon}\ {\isachardoublequote}{\isacharparenleft}nat\ {\isasymRightarrow}\ bool{\isacharparenright}\ {\isasymRightarrow}\ nat\ T{\isachardoublequote}\isanewline
\ \ {\isachardoublequote}sat\ p\ {\isasymequiv}\ do\ {\isacharbraceleft}x{\isasymleftarrow}item{\isacharsemicolon}\ if\ p\ x\ then\ ret\ x\ else\ fail{\isacharbraceright}{\isachardoublequote}\isanewline
\ \ digitp\ \ \ \ \ \ \ {\isacharcolon}{\isacharcolon}\ {\isachardoublequote}nat\ T{\isachardoublequote}\isanewline
\ \ {\isachardoublequote}digitp\ {\isasymequiv}\ sat\ {\isacharparenleft}{\isasymlambda}x{\isachardot}\ x\ {\isacharless}\ {\isadigit{1}}{\isadigit{0}}{\isacharparenright}{\isachardoublequote}\isamarkupfalse%
%
\begin{isamarkuptext}%
The intended semantics of \isa{many} is that it maps a parser $p$ into one
  that applies $p$ as often as possible and collects the results (which may be 
  none). \isa{many{\isadigit{1}}} requires at least one successful run of $p$.%
\end{isamarkuptext}%
\isamarkuptrue%
\isacommand{consts}\isanewline
many\ \ {\isacharcolon}{\isacharcolon}\ {\isachardoublequote}{\isacharprime}a\ T\ {\isasymRightarrow}\ {\isacharprime}a\ list\ T{\isachardoublequote}\isanewline
many{\isadigit{1}}\ {\isacharcolon}{\isacharcolon}\ {\isachardoublequote}{\isacharprime}a\ T\ {\isasymRightarrow}\ {\isacharprime}a\ list\ T{\isachardoublequote}\isamarkupfalse%
%
\begin{isamarkuptext}%
We cannot define \isa{many}, since it is not primitive recursive 
  and there is no termination measure. 
  \label{isa:many-unfold}%
\end{isamarkuptext}%
\isamarkuptrue%
\isacommand{axioms}\isanewline
many{\isacharunderscore}unfold{\isacharcolon}\ {\isachardoublequote}many\ p\ {\isacharequal}\ {\isacharparenleft}{\isacharparenleft}do\ {\isacharbraceleft}x\ {\isasymleftarrow}\ p{\isacharsemicolon}\ xs\ {\isasymleftarrow}\ many\ p{\isacharsemicolon}\ ret\ {\isacharparenleft}x{\isacharhash}xs{\isacharparenright}{\isacharbraceright}{\isacharparenright}\ {\isasymparallel}\ ret\ {\isacharbrackleft}{\isacharbrackright}{\isacharparenright}{\isachardoublequote}\isanewline
\isanewline
\isamarkupfalse%
\isacommand{defs}\isanewline
many{\isadigit{1}}{\isacharunderscore}def{\isacharcolon}\ {\isachardoublequote}many{\isadigit{1}}\ p\ {\isasymequiv}\ {\isacharparenleft}do\ {\isacharbraceleft}x\ {\isasymleftarrow}\ p{\isacharsemicolon}\ xs\ {\isasymleftarrow}\ many\ p{\isacharsemicolon}\ ret\ {\isacharparenleft}x{\isacharhash}xs{\isacharparenright}{\isacharbraceright}{\isacharparenright}{\isachardoublequote}\isamarkupfalse%
%
\begin{isamarkuptext}%
This is the most convenient and expressive rule we can hope for at the
   moment.%
\end{isamarkuptext}%
\isamarkuptrue%
\isacommand{lemma}\ many{\isacharunderscore}step{\isacharcolon}\ {\isachardoublequote}{\isasymlbrakk}\ {\isasymturnstile}\ {\isasymlangle}{\isacharparenleft}do\ {\isacharbraceleft}x\ {\isasymleftarrow}\ p{\isacharsemicolon}\ xs\ {\isasymleftarrow}\ many\ p{\isacharsemicolon}\ ret\ {\isacharparenleft}x{\isacharhash}xs{\isacharparenright}{\isacharbraceright}{\isacharparenright}{\isasymrangle}P\ {\isasymor}\isactrlsub D\ \isanewline
\ \ \ \ \ \ \ \ \ \ \ \ \ \ \ \ \ \ \ \ \ \ {\isasymlangle}ret\ {\isacharbrackleft}{\isacharbrackright}{\isasymrangle}P\ {\isasymand}\isactrlsub D\ {\isacharbrackleft}{\isacharhash}\ x{\isasymleftarrow}p{\isacharbrackright}{\isacharparenleft}Ret\ False{\isacharparenright}\ {\isasymrbrakk}\ {\isasymLongrightarrow}\ {\isasymturnstile}\ {\isasymlangle}many\ p{\isasymrangle}P{\isachardoublequote}\isamarkupfalse%
\isamarkupfalse%
\isamarkupfalse%
\isamarkupfalse%
\isamarkupfalse%
\isamarkupfalse%
\isamarkupfalse%
\isamarkupfalse%
\isamarkupfalse%
\isamarkupfalse%
\isamarkupfalse%
\isamarkupfalse%
\isamarkupfalse%
\isamarkupfalse%
\isamarkupfalse%
\isamarkupfalse%
\isamarkupfalse%
\isamarkupfalse%
\isamarkupfalse%
\isamarkupfalse%
\isamarkupfalse%
\isamarkupfalse%
\isamarkupfalse%
\isamarkupfalse%
\isamarkupfalse%
\isamarkupfalse%
\isamarkupfalse%
\isamarkupfalse%
\isamarkupfalse%
\isamarkupfalse%
\isamarkupfalse%
\isanewline
\isamarkupfalse%
\isacommand{constdefs}\isanewline
natp\ {\isacharcolon}{\isacharcolon}\ {\isachardoublequote}nat\ T{\isachardoublequote}\isanewline
{\isachardoublequote}natp\ {\isasymequiv}\ do\ {\isacharbraceleft}ns\ {\isasymleftarrow}\ many{\isadigit{1}}\ digitp{\isacharsemicolon}\ ret\ {\isacharparenleft}foldl\ {\isacharparenleft}{\isasymlambda}r\ n{\isachardot}\ {\isadigit{1}}{\isadigit{0}}\ {\isacharasterisk}\ r\ {\isacharplus}\ n{\isacharparenright}\ {\isadigit{0}}\ ns{\isacharparenright}{\isacharbraceright}{\isachardoublequote}\isamarkupfalse%
%
\begin{isamarkuptext}%
The parser for natural numbers \isa{natp} works on an input stream
  that conists of natural numbers and reads numbers between 0 and 9 (inclusive) until 
  no such number can be read. Then it transforms its result list into a number
  by folding an appropriate function into the list. Of course, one might just as
  well consider an input stream of bounded numbers (e.g. ASCII characters in their
  numeric representation) and then read `0' to `9', but this would not 
  provide any interesting further insight.%
\end{isamarkuptext}%
\isamarkuptrue%
%
\isamarkupsubsection{Auxiliary Lemmas%
}
\isamarkuptrue%
%
\begin{isamarkuptext}%
A convenient rendition of axiom \isa{altD{\isacharunderscore}iff} as a rule.%
\end{isamarkuptext}%
\isamarkuptrue%
\isacommand{lemma}\ altD{\isacharunderscore}iff{\isacharunderscore}lifted{\isadigit{1}}{\isacharcolon}\ {\isachardoublequote}{\isasymlbrakk}{\isasymturnstile}\ A\ {\isasymlongrightarrow}\isactrlsub D\ {\isasymlangle}x{\isasymleftarrow}q{\isasymrangle}{\isacharparenleft}P\ x{\isacharparenright}{\isacharsemicolon}\ {\isasymturnstile}\ A\ {\isasymlongrightarrow}\isactrlsub D\ {\isacharbrackleft}{\isacharhash}\ x{\isasymleftarrow}p{\isacharbrackright}{\isacharparenleft}Ret\ False{\isacharparenright}{\isasymrbrakk}\ {\isasymLongrightarrow}\ {\isasymturnstile}\ A\ {\isasymlongrightarrow}\isactrlsub D\ {\isasymlangle}x{\isasymleftarrow}\ p{\isasymparallel}q{\isasymrangle}{\isacharparenleft}P\ x{\isacharparenright}{\isachardoublequote}\isanewline
\isamarkupfalse%
\isacommand{proof}\ {\isacharminus}\ \isanewline
\ \ \isamarkupfalse%
\isacommand{have}\ {\isachardoublequote}{\isasymturnstile}\ {\isacharparenleft}{\isasymlangle}x{\isasymleftarrow}p{\isasymparallel}q{\isasymrangle}{\isacharparenleft}P\ x{\isacharparenright}\ {\isasymlongleftrightarrow}\isactrlsub D\ {\isasymlangle}x{\isasymleftarrow}p{\isasymrangle}{\isacharparenleft}P\ x{\isacharparenright}\ {\isasymor}\isactrlsub D\ {\isasymlangle}x{\isasymleftarrow}q{\isasymrangle}{\isacharparenleft}P\ x{\isacharparenright}\ {\isasymand}\isactrlsub D\ {\isacharbrackleft}{\isacharhash}\ x{\isasymleftarrow}p{\isacharbrackright}{\isacharparenleft}Ret\ False{\isacharparenright}{\isacharparenright}\ {\isasymlongrightarrow}\isactrlsub D\isanewline
\ \ \ \ \ \ \ \ \ \ {\isacharparenleft}A\ {\isasymlongrightarrow}\isactrlsub D\ {\isasymlangle}x{\isasymleftarrow}q{\isasymrangle}{\isacharparenleft}P\ x{\isacharparenright}{\isacharparenright}\ {\isasymlongrightarrow}\isactrlsub D\ {\isacharparenleft}A\ {\isasymlongrightarrow}\isactrlsub D\ {\isacharbrackleft}{\isacharhash}\ x{\isasymleftarrow}p{\isacharbrackright}{\isacharparenleft}Ret\ False{\isacharparenright}{\isacharparenright}\ {\isasymlongrightarrow}\isactrlsub D\isanewline
\ \ \ \ \ \ \ \ \ \ \ A\ {\isasymlongrightarrow}\isactrlsub D\ {\isasymlangle}x{\isasymleftarrow}\ p{\isasymparallel}q{\isasymrangle}{\isacharparenleft}P\ x{\isacharparenright}{\isachardoublequote}\isanewline
\ \ \ \ \isamarkupfalse%
\isacommand{by}\ {\isacharparenleft}simp\ add{\isacharcolon}\ pdl{\isacharunderscore}taut{\isacharparenright}\isanewline
\ \ \isamarkupfalse%
\isacommand{moreover}\ \isanewline
\ \ \isamarkupfalse%
\isacommand{note}\ altD{\isacharunderscore}iff\isanewline
\ \ \isamarkupfalse%
\isacommand{moreover}\isanewline
\ \ \isamarkupfalse%
\isacommand{assume}\ {\isachardoublequote}{\isasymturnstile}\ A\ {\isasymlongrightarrow}\isactrlsub D\ {\isasymlangle}x{\isasymleftarrow}q{\isasymrangle}{\isacharparenleft}P\ x{\isacharparenright}{\isachardoublequote}\isanewline
\ \ \isamarkupfalse%
\isacommand{moreover}\isanewline
\ \ \isamarkupfalse%
\isacommand{assume}\ {\isachardoublequote}{\isasymturnstile}\ A\ {\isasymlongrightarrow}\isactrlsub D\ {\isacharbrackleft}{\isacharhash}\ x{\isasymleftarrow}p{\isacharbrackright}{\isacharparenleft}Ret\ False{\isacharparenright}{\isachardoublequote}\isanewline
\ \ \isamarkupfalse%
\isacommand{ultimately}\isanewline
\ \ \isamarkupfalse%
\isacommand{show}\ {\isacharquery}thesis\ \isamarkupfalse%
\isacommand{by}\ {\isacharparenleft}rule\ pdl{\isacharunderscore}mp{\isacharunderscore}{\isadigit{3}}x{\isacharparenright}\isanewline
\isamarkupfalse%
\isacommand{qed}\isamarkupfalse%
%
\begin{isamarkuptext}%
The correctness of \isa{natp} obviously relies on the correctness of \isa{digitp}, 
  which is proved first.%
\end{isamarkuptext}%
\isamarkuptrue%
\isacommand{theorem}\ digitp{\isacharunderscore}nat{\isacharcolon}\ {\isachardoublequote}{\isasymturnstile}\ GetInput\ {\isacharequal}\isactrlsub D\ Ret\ {\isacharparenleft}{\isadigit{1}}{\isacharhash}ys{\isacharparenright}\ {\isasymlongrightarrow}\isactrlsub D\ {\isasymlangle}x{\isasymleftarrow}digitp{\isasymrangle}{\isacharparenleft}Ret\ {\isacharparenleft}x\ {\isacharequal}\ {\isadigit{1}}{\isacharparenright}\ {\isasymand}\isactrlsub D\ GetInput\ {\isacharequal}\isactrlsub D\ Ret\ ys{\isacharparenright}{\isachardoublequote}\isanewline
\ \ {\isacharparenleft}\isakeyword{is}\ {\isachardoublequote}{\isasymturnstile}\ {\isacharquery}A\ {\isasymlongrightarrow}\isactrlsub D\ {\isasymlangle}digitp{\isasymrangle}{\isacharparenleft}{\isasymlambda}x{\isachardot}\ {\isacharquery}C\ x\ {\isasymand}\isactrlsub D\ {\isacharquery}D{\isacharparenright}{\isachardoublequote}{\isacharparenright}\isanewline
\ \ \isamarkupfalse%
\isacommand{apply}{\isacharparenleft}unfold\ digitp{\isacharunderscore}def\ sat{\isacharunderscore}def{\isacharparenright}\isanewline
\ \ \isamarkupfalse%
\isacommand{apply}{\isacharparenleft}rule\ pdl{\isacharunderscore}plugD{\isacharunderscore}lifted{\isadigit{1}}{\isacharparenright}\isanewline
\ \ \isamarkupfalse%
\isacommand{apply}{\isacharparenleft}rule\ get{\isacharunderscore}item{\isacharparenright}\isanewline
\ \ \isamarkupfalse%
\isacommand{apply}{\isacharparenleft}rule\ allI{\isacharparenright}\isanewline
\ \ \isamarkupfalse%
\isacommand{apply}{\isacharparenleft}simp\ add{\isacharcolon}\ split{\isacharunderscore}if{\isacharparenright}\isanewline
\ \ \isamarkupfalse%
\isacommand{apply}{\isacharparenleft}safe{\isacharparenright}\ \isanewline
\ \ \isamarkupfalse%
\isacommand{apply}{\isacharparenleft}rule\ pdl{\isacharunderscore}iffD{\isadigit{2}}{\isacharbrackleft}OF\ pdl{\isacharunderscore}retD{\isacharbrackright}{\isacharparenright}\isanewline
\ \ \isamarkupfalse%
\isacommand{by}\ {\isacharparenleft}simp\ add{\isacharcolon}\ pdl{\isacharunderscore}taut{\isacharparenright}\ %
\isamarkupcmt{For the else-branch we obtain a contradiction, since the input was 1%
}
\isamarkupfalse%
%
\begin{isamarkuptext}%
On empty input, \isa{digitp} will fail.%
\end{isamarkuptext}%
\isamarkuptrue%
\isacommand{theorem}\ digitp{\isacharunderscore}fail{\isacharcolon}\ {\isachardoublequote}{\isasymturnstile}\ GetInput\ {\isacharequal}\isactrlsub D\ Ret\ {\isacharbrackleft}{\isacharbrackright}\ {\isasymlongrightarrow}\isactrlsub D\ {\isacharbrackleft}{\isacharhash}\ digitp{\isacharbrackright}{\isacharparenleft}{\isasymlambda}x{\isachardot}\ Ret\ False{\isacharparenright}{\isachardoublequote}\isanewline
\ \ \isamarkupfalse%
\isacommand{apply}{\isacharparenleft}simp\ add{\isacharcolon}\ digitp{\isacharunderscore}def\ sat{\isacharunderscore}def{\isacharparenright}\isanewline
\ \ \isamarkupfalse%
\isacommand{apply}{\isacharparenleft}rule\ pdl{\isacharunderscore}plugB{\isacharunderscore}lifted{\isadigit{1}}{\isacharparenright}\isanewline
\ \ \isamarkupfalse%
\isacommand{apply}{\isacharparenleft}rule\ GetInput{\isacharunderscore}item{\isacharunderscore}fail{\isacharparenright}\isanewline
\ \ \isamarkupfalse%
\isacommand{apply}{\isacharparenleft}rule\ allI{\isacharparenright}\isanewline
\ \ \isamarkupfalse%
\isacommand{apply}{\isacharparenleft}rule\ pdl{\isacharunderscore}False{\isacharunderscore}imp{\isacharparenright}\isanewline
\isamarkupfalse%
\isacommand{done}\isanewline
\isanewline
\isanewline
\isamarkupfalse%
\isacommand{lemma}\ ret{\isacharunderscore}nil{\isacharunderscore}aux{\isacharcolon}\ {\isachardoublequote}\ {\isasymturnstile}\ A\ {\isasymand}\isactrlsub D\ B\ {\isasymlongrightarrow}\isactrlsub D\isanewline
\ \ {\isasymlangle}ret\ {\isacharbrackleft}{\isacharbrackright}{\isasymrangle}{\isacharparenleft}{\isasymlambda}xs{\isachardot}\ A\ {\isasymand}\isactrlsub D\ B\ {\isasymand}\isactrlsub D\ Ret\ {\isacharparenleft}xs\ {\isacharequal}\ {\isacharbrackleft}{\isacharbrackright}{\isacharparenright}{\isacharparenright}{\isachardoublequote}\isamarkupfalse%
\isamarkupfalse%
\isamarkupfalse%
\isanewline
\isanewline
\isamarkupfalse%
\isacommand{lemma}\ ret{\isacharunderscore}one{\isacharunderscore}aux{\isacharcolon}\ {\isachardoublequote}{\isasymturnstile}\ A\ {\isasymlongrightarrow}\isactrlsub D\ \isanewline
\ \ \ \ \ \ \ \ \ \ \ \ \ \ \ \ \ \ \ \ \ \ {\isasymlangle}ret\ {\isacharparenleft}Suc\ {\isadigit{0}}{\isacharparenright}{\isasymrangle}{\isacharparenleft}{\isasymlambda}n{\isachardot}\ Ret\ {\isacharparenleft}n\ {\isacharequal}\ Suc\ {\isadigit{0}}{\isacharparenright}\ {\isasymand}\isactrlsub D\ A{\isacharparenright}{\isachardoublequote}\isamarkupfalse%
\isamarkupfalse%
\isamarkupfalse%
\isanewline
\isanewline
\isanewline
\isamarkupfalse%
\isacommand{lemma}\ pdl{\isacharunderscore}eqD{\isacharunderscore}aux{\isadigit{1}}{\isacharcolon}\ {\isachardoublequote}{\isasymturnstile}\ {\isacharparenleft}B\ {\isasymand}\isactrlsub D\ C\ {\isasymlongrightarrow}\isactrlsub D\ {\isasymlangle}p\ b{\isasymrangle}P{\isacharparenright}\ {\isasymlongrightarrow}\isactrlsub D\ Ret\ {\isacharparenleft}a\ {\isacharequal}\ b{\isacharparenright}\ {\isasymand}\isactrlsub D\ B\ {\isasymand}\isactrlsub D\ C\ {\isasymlongrightarrow}\isactrlsub D\ {\isasymlangle}p\ a{\isasymrangle}P{\isachardoublequote}\isamarkupfalse%
\isamarkupfalse%
\isamarkupfalse%
\isamarkupfalse%
\isamarkupfalse%
\isamarkupfalse%
\isamarkupfalse%
\isamarkupfalse%
\isamarkupfalse%
\isamarkupfalse%
\isamarkupfalse%
\isamarkupfalse%
\isamarkupfalse%
\isamarkupfalse%
\isamarkupfalse%
\isanewline
\isamarkupfalse%
\isacommand{lemma}\ pdl{\isacharunderscore}eqD{\isacharunderscore}aux{\isadigit{2}}{\isacharcolon}\ {\isachardoublequote}{\isasymturnstile}\ {\isacharparenleft}A\ {\isasymlongrightarrow}\isactrlsub D\ {\isasymlangle}\ p\ b{\isasymrangle}P{\isacharparenright}\ {\isasymlongrightarrow}\isactrlsub D\ A\ {\isasymand}\isactrlsub D\ Ret\ {\isacharparenleft}a\ {\isacharequal}\ b{\isacharparenright}\ {\isasymlongrightarrow}\isactrlsub D\ {\isasymlangle}\ p\ a{\isasymrangle}P{\isachardoublequote}\isamarkupfalse%
\isamarkupfalse%
\isamarkupfalse%
\isamarkupfalse%
\isamarkupfalse%
\isamarkupfalse%
\isamarkupfalse%
\isamarkupfalse%
\isamarkupfalse%
\isamarkupfalse%
\isamarkupfalse%
\isamarkupfalse%
\isamarkupfalse%
\isamarkupfalse%
\isamarkupfalse%
\isanewline
\isanewline
\isamarkupfalse%
\isacommand{lemma}\ pdl{\isacharunderscore}imp{\isacharunderscore}strg{\isadigit{1}}{\isacharcolon}\ {\isachardoublequote}{\isasymturnstile}\ A\ {\isasymlongrightarrow}\isactrlsub D\ C\ {\isasymLongrightarrow}\ {\isasymturnstile}\ A\ {\isasymand}\isactrlsub D\ B\ {\isasymlongrightarrow}\isactrlsub D\ C{\isachardoublequote}\isamarkupfalse%
\isamarkupfalse%
\isamarkupfalse%
\isamarkupfalse%
\isamarkupfalse%
\isamarkupfalse%
\isamarkupfalse%
\isanewline
\isamarkupfalse%
\isacommand{lemma}\ pdl{\isacharunderscore}imp{\isacharunderscore}strg{\isadigit{2}}{\isacharcolon}\ {\isachardoublequote}{\isasymturnstile}\ B\ {\isasymlongrightarrow}\isactrlsub D\ C\ {\isasymLongrightarrow}\ {\isasymturnstile}\ A\ {\isasymand}\isactrlsub D\ B\ {\isasymlongrightarrow}\isactrlsub D\ C{\isachardoublequote}\isamarkupfalse%
\isamarkupfalse%
\isamarkupfalse%
\isamarkupfalse%
\isamarkupfalse%
\isamarkupfalse%
\isamarkupfalse%
\isamarkupfalse%
%
\isamarkupsubsection{Correctness of the Monadic Parser%
}
\isamarkuptrue%
%
\begin{isamarkuptext}%
The following is a major theorem, more because of its complexity and since it 
  involves most of the axioms given for the monad, than because of its
  theoretical insight. Essentially, it states that \isa{natp} behaves
  totally correct for a given input.
  \label{isa:natp-proof}%
\end{isamarkuptext}%
\isamarkuptrue%
\isacommand{theorem}\ natp{\isacharunderscore}corr{\isacharcolon}\ {\isachardoublequote}{\isasymturnstile}\ {\isasymlangle}do\ {\isacharbraceleft}uu{\isasymleftarrow}setInput\ {\isacharbrackleft}{\isadigit{1}}{\isacharbrackright}{\isacharsemicolon}\ natp{\isacharbraceright}{\isasymrangle}{\isacharparenleft}{\isasymlambda}n{\isachardot}\ Ret\ {\isacharparenleft}n\ {\isacharequal}\ {\isadigit{1}}{\isacharparenright}\ {\isasymand}\isactrlsub D\ Eot{\isacharparenright}{\isachardoublequote}\isanewline
\isamarkupfalse%
\isacommand{proof}\ {\isacharminus}\isanewline
\ \ \isamarkupfalse%
\isacommand{have}\ {\isachardoublequote}{\isasymturnstile}\ {\isasymlangle}uu{\isasymleftarrow}setInput\ {\isacharbrackleft}{\isadigit{1}}{\isacharbrackright}{\isasymrangle}{\isacharparenleft}GetInput\ {\isacharequal}\isactrlsub D\ Ret\ {\isacharbrackleft}{\isadigit{1}}{\isacharbrackright}{\isacharparenright}{\isachardoublequote}\isanewline
\ \ \ \ \isamarkupfalse%
\isacommand{by}\ {\isacharparenleft}rule\ set{\isacharunderscore}get{\isacharparenright}\isanewline
\ \ \isamarkupfalse%
\isacommand{moreover}\isanewline
\ \ \isamarkupfalse%
\isacommand{have}\ {\isachardoublequote}{\isasymforall}uu{\isacharcolon}{\isacharcolon}unit{\isachardot}\ {\isasymturnstile}\ GetInput\ {\isacharequal}\isactrlsub D\ Ret\ {\isacharbrackleft}{\isadigit{1}}{\isacharbrackright}\ {\isasymlongrightarrow}\isactrlsub D\ {\isasymlangle}n{\isasymleftarrow}natp{\isasymrangle}{\isacharparenleft}Ret\ {\isacharparenleft}n\ {\isacharequal}\ {\isadigit{1}}{\isacharparenright}\ {\isasymand}\isactrlsub D\ Eot{\isacharparenright}{\isachardoublequote}\isanewline
\ \ \isamarkupfalse%
\isacommand{proof}\isanewline
\ \ \ \ \isamarkupfalse%
\isacommand{fix}\ uu\isanewline
\ \ \ \ %
\isamarkupcmt{The actual proof starts here: from a given input, show that \isa{natp} is correct%
}
\isanewline
\ \ \ \ \isamarkupfalse%
\isacommand{show}\ {\isachardoublequote}{\isasymturnstile}\ GetInput\ {\isacharequal}\isactrlsub D\ Ret\ {\isacharbrackleft}{\isadigit{1}}{\isacharbrackright}\ {\isasymlongrightarrow}\isactrlsub D\ {\isasymlangle}natp{\isasymrangle}{\isacharparenleft}{\isasymlambda}n{\isachardot}\ Ret\ {\isacharparenleft}n\ {\isacharequal}\ {\isadigit{1}}{\isacharparenright}\ {\isasymand}\isactrlsub D\ Eot{\isacharparenright}{\isachardoublequote}\isanewline
\ \ \ \ \isamarkupfalse%
\isacommand{proof}\ {\isacharminus}\isanewline
\ \ \ \ \ \ %
\isamarkupcmt{Prove the formula with defn. of \isa{natp} unfolded%
}
\isanewline
\ \ \ \ \ \ \isamarkupfalse%
\isacommand{have}\ {\isachardoublequote}{\isasymturnstile}\ GetInput\ {\isacharequal}\isactrlsub D\ Ret\ {\isacharbrackleft}{\isadigit{1}}{\isacharbrackright}\ {\isasymlongrightarrow}\isactrlsub D\ {\isasymlangle}do\ {\isacharbraceleft}x{\isasymleftarrow}digitp{\isacharsemicolon}\ xs{\isasymleftarrow}many\ digitp{\isacharsemicolon}\ ret\ {\isacharparenleft}foldl\ {\isacharparenleft}{\isasymlambda}r{\isachardot}\ op\ {\isacharplus}\ {\isacharparenleft}{\isadigit{1}}{\isadigit{0}}\ {\isacharasterisk}\ r{\isacharparenright}{\isacharparenright}\ x\ xs{\isacharparenright}{\isacharbraceright}{\isasymrangle}{\isacharparenleft}{\isasymlambda}n{\isachardot}\ Ret\ {\isacharparenleft}n\ {\isacharequal}\ {\isadigit{1}}{\isacharparenright}\ {\isasymand}\isactrlsub D\ Eot{\isacharparenright}{\isachardoublequote}\ {\isacharparenleft}\isakeyword{is}\ {\isachardoublequote}{\isasymturnstile}\ {\isacharquery}a\ {\isasymlongrightarrow}\isactrlsub D\ {\isacharquery}b{\isachardoublequote}{\isacharparenright}\isanewline
\ \ \ \ \ \ \isamarkupfalse%
\isacommand{proof}\ {\isacharminus}\ %
\isamarkupcmt{Work out each atomic program separately%
}
\isanewline
\ \ \ \ \ \ \ \ \isamarkupfalse%
\isacommand{have}\ {\isachardoublequote}{\isasymturnstile}\ GetInput\ {\isacharequal}\isactrlsub D\ Ret\ {\isacharbrackleft}{\isadigit{1}}{\isacharbrackright}\ {\isasymlongrightarrow}\isactrlsub D\ {\isasymlangle}x{\isasymleftarrow}digitp{\isasymrangle}{\isacharparenleft}Ret\ {\isacharparenleft}x{\isacharequal}{\isadigit{1}}{\isacharparenright}\ {\isasymand}\isactrlsub D\ GetInput\ {\isacharequal}\isactrlsub D\ Ret\ {\isacharbrackleft}{\isacharbrackright}{\isacharparenright}{\isachardoublequote}\isanewline
\ \ \ \ \ \ \ \ \ \ \isamarkupfalse%
\isacommand{by}\ {\isacharparenleft}rule\ digitp{\isacharunderscore}nat{\isacharparenright}\isanewline
\ \ \ \ \ \ \ \ \isamarkupfalse%
\isacommand{moreover}\isanewline
\ \ \ \ \ \ \ \ \isamarkupfalse%
\isacommand{have}\ {\isachardoublequote}{\isasymforall}x{\isachardot}\ {\isasymturnstile}\ {\isacharparenleft}Ret\ {\isacharparenleft}x{\isacharequal}{\isacharparenleft}{\isadigit{1}}{\isacharcolon}{\isacharcolon}nat{\isacharparenright}{\isacharparenright}\ {\isasymand}\isactrlsub D\ GetInput\ {\isacharequal}\isactrlsub D\ Ret\ {\isacharbrackleft}{\isacharbrackright}{\isacharparenright}\ {\isasymlongrightarrow}\isactrlsub D\isanewline
\ \ \ \ \ \ \ \ \ \ \ \ {\isacharparenleft}{\isasymlangle}do\ {\isacharbraceleft}xs{\isasymleftarrow}many\ digitp{\isacharsemicolon}\ ret\ {\isacharparenleft}foldl\ {\isacharparenleft}{\isasymlambda}r{\isachardot}\ op\ {\isacharplus}\ {\isacharparenleft}{\isadigit{1}}{\isadigit{0}}\ {\isacharasterisk}\ r{\isacharparenright}{\isacharparenright}\ x\ xs{\isacharparenright}{\isacharbraceright}{\isasymrangle}{\isacharparenleft}{\isasymlambda}n{\isachardot}\ Ret{\isacharparenleft}n{\isacharequal}{\isadigit{1}}{\isacharparenright}\ {\isasymand}\isactrlsub D\ Eot{\isacharparenright}{\isacharparenright}{\isachardoublequote}\isanewline
\ \ \ \ \ \ \ \ \isamarkupfalse%
\isacommand{proof}\ %
\isamarkupcmt{Here, \isa{digitp} will fail, ie. \isa{many} will return \isa{{\isacharbrackleft}{\isacharbrackright}}%
}
\isanewline
\ \ \ \ \ \ \ \ \ \ \isamarkupfalse%
\isacommand{fix}\ x\isanewline
\ \ \ \ \ \ \ \ \ \ \isamarkupfalse%
\isacommand{show}\ {\isachardoublequote}{\isasymturnstile}\ Ret\ {\isacharparenleft}x\ {\isacharequal}\ {\isadigit{1}}{\isacharparenright}\ {\isasymand}\isactrlsub D\ GetInput\ {\isacharequal}\isactrlsub D\ Ret\ {\isacharbrackleft}{\isacharbrackright}\ {\isasymlongrightarrow}\isactrlsub D\isanewline
\ \ \ \ \ \ \ \ \ \ \ \ \ {\isasymlangle}do\ {\isacharbraceleft}xs{\isasymleftarrow}many\ digitp{\isacharsemicolon}\ ret\ {\isacharparenleft}foldl\ {\isacharparenleft}{\isasymlambda}r{\isachardot}\ op\ {\isacharplus}\ {\isacharparenleft}{\isadigit{1}}{\isadigit{0}}\ {\isacharasterisk}\ r{\isacharparenright}{\isacharparenright}\ x\ xs{\isacharparenright}{\isacharbraceright}{\isasymrangle}{\isacharparenleft}{\isasymlambda}n{\isachardot}\ Ret\ {\isacharparenleft}n\ {\isacharequal}\ {\isadigit{1}}{\isacharparenright}\ {\isasymand}\isactrlsub D\ Eot{\isacharparenright}{\isachardoublequote}\isanewline
\ \ \ \ \ \ \ \ \ \ \isamarkupfalse%
\isacommand{proof}\ {\isacharparenleft}rule\ pdl{\isacharunderscore}plugD{\isacharunderscore}lifted{\isadigit{1}}{\isacharbrackleft}\isakeyword{where}\ B\ {\isacharequal}\ {\isachardoublequote}{\isasymlambda}xs{\isachardot}\ Ret\ {\isacharparenleft}x\ {\isacharequal}\ {\isadigit{1}}{\isacharparenright}\ {\isasymand}\isactrlsub D\ GetInput\ {\isacharequal}\isactrlsub D\ Ret\ {\isacharbrackleft}{\isacharbrackright}\ {\isasymand}\isactrlsub D\isanewline
\ \ \ \ \ \ \ \ \ \ \ \ \ \ \ \ \ \ \ \ \ \ \ \ \ \ \ \ \ \ \ \ \ \ \ \ \ \ \ \ \ \ \ \ \ \ \ \ \ \ \ \ \ \ \ \ Ret\ {\isacharparenleft}xs\ {\isacharequal}\ {\isacharbrackleft}{\isacharbrackright}{\isacharparenright}{\isachardoublequote}{\isacharbrackright}{\isacharparenright}\isanewline
\ \ \ \ \ \ \ \ \ \ \ \ \isamarkupfalse%
\isacommand{show}\ {\isachardoublequote}{\isasymturnstile}\ Ret\ {\isacharparenleft}x\ {\isacharequal}\ {\isadigit{1}}{\isacharparenright}\ {\isasymand}\isactrlsub D\ GetInput\ {\isacharequal}\isactrlsub D\ Ret\ {\isacharbrackleft}{\isacharbrackright}\ {\isasymlongrightarrow}\isactrlsub D\isanewline
\ \ \ \ \ \ \ \ \ \ \ \ \ \ {\isasymlangle}many\ digitp{\isasymrangle}{\isacharparenleft}{\isasymlambda}xs{\isachardot}\ Ret\ {\isacharparenleft}x\ {\isacharequal}\ {\isadigit{1}}{\isacharparenright}\ {\isasymand}\isactrlsub D\ GetInput\ {\isacharequal}\isactrlsub D\ Ret\ {\isacharbrackleft}{\isacharbrackright}\ {\isasymand}\isactrlsub D\ Ret\ {\isacharparenleft}xs\ {\isacharequal}\ {\isacharbrackleft}{\isacharbrackright}{\isacharparenright}{\isacharparenright}{\isachardoublequote}\isanewline
\ \ \ \ \ \ \ \ \ \ \ \ \ \ \isamarkupfalse%
\isacommand{apply}{\isacharparenleft}subst\ many{\isacharunderscore}unfold{\isacharparenright}\isanewline
\ \ \ \ \ \ \ \ \ \ \ \ \ \ \isamarkupfalse%
\isacommand{apply}{\isacharparenleft}rule\ altD{\isacharunderscore}iff{\isacharunderscore}lifted{\isadigit{1}}{\isacharparenright}\isanewline
\ \ \ \ \ \ \ \ \ \ \ \ \ \ \isamarkupfalse%
\isacommand{apply}{\isacharparenleft}rule\ ret{\isacharunderscore}nil{\isacharunderscore}aux{\isacharparenright}\isanewline
\ \ \ \ \ \ \ \ \ \ \ \ \ \ \isamarkupfalse%
\isacommand{apply}{\isacharparenleft}rule\ pdl{\isacharunderscore}plugB{\isacharunderscore}lifted{\isadigit{1}}{\isacharparenright}\isanewline
\ \ \ \ \ \ \ \ \ \ \ \ \ \ \isamarkupfalse%
\isacommand{apply}{\isacharparenleft}rule\ pdl{\isacharunderscore}imp{\isacharunderscore}strg{\isadigit{2}}{\isacharparenright}\isanewline
\ \ \ \ \ \ \ \ \ \ \ \ \ \ \isamarkupfalse%
\isacommand{apply}{\isacharparenleft}rule\ digitp{\isacharunderscore}fail{\isacharparenright}\isanewline
\ \ \ \ \ \ \ \ \ \ \ \ \ \ \isamarkupfalse%
\isacommand{apply}{\isacharparenleft}rule\ allI{\isacharparenright}\ \isanewline
\ \ \ \ \ \ \ \ \ \ \ \ \ \ \isamarkupfalse%
\isacommand{by}\ {\isacharparenleft}simp\ add{\isacharcolon}\ pdl{\isacharunderscore}taut{\isacharparenright}\isanewline
\ \ \ \ \ \ \ \ \ \ \isamarkupfalse%
\isacommand{next}\isanewline
\ \ \ \ \ \ \ \ \ \ \ \ \isamarkupfalse%
\isacommand{show}\ {\isachardoublequote}{\isasymforall}xs{\isachardot}\ {\isasymturnstile}\ Ret\ {\isacharparenleft}x\ {\isacharequal}\ {\isadigit{1}}{\isacharparenright}\ {\isasymand}\isactrlsub D\ GetInput\ {\isacharequal}\isactrlsub D\ Ret\ {\isacharbrackleft}{\isacharbrackright}\ {\isasymand}\isactrlsub D\ Ret\ {\isacharparenleft}xs\ {\isacharequal}\ {\isacharbrackleft}{\isacharbrackright}{\isacharparenright}\ {\isasymlongrightarrow}\isactrlsub D\isanewline
\ \ \ \ \ \ \ \ \ \ \ \ \ \ \ \ \ \ \ \ \ \ \ \ \ {\isasymlangle}ret\ {\isacharparenleft}foldl\ {\isacharparenleft}{\isasymlambda}r{\isachardot}\ op\ {\isacharplus}\ {\isacharparenleft}{\isadigit{1}}{\isadigit{0}}\ {\isacharasterisk}\ r{\isacharparenright}{\isacharparenright}\ x\ xs{\isacharparenright}{\isasymrangle}{\isacharparenleft}{\isasymlambda}n{\isachardot}\ Ret\ {\isacharparenleft}n\ {\isacharequal}\ {\isadigit{1}}{\isacharparenright}\ {\isasymand}\isactrlsub D\ Eot{\isacharparenright}{\isachardoublequote}\isanewline
\ \ \ \ \ \ \ \ \ \ \ \ \ \ \isamarkupfalse%
\isacommand{apply}{\isacharparenleft}rule\ allI{\isacharparenright}\isanewline
\ \ \ \ \ \ \ \ \ \ \ \ \ \ \isamarkupfalse%
\isacommand{apply}{\isacharparenleft}rule\ pdl{\isacharunderscore}eqD{\isacharunderscore}aux{\isadigit{1}}{\isacharbrackleft}THEN\ pdl{\isacharunderscore}mp{\isacharbrackright}{\isacharparenright}\isanewline
\ \ \ \ \ \ \ \ \ \ \ \ \ \ \isamarkupfalse%
\isacommand{apply}{\isacharparenleft}rule\ pdl{\isacharunderscore}eqD{\isacharunderscore}aux{\isadigit{2}}{\isacharbrackleft}THEN\ pdl{\isacharunderscore}mp{\isacharbrackright}{\isacharparenright}\isanewline
\ \ \ \ \ \ \ \ \ \ \ \ \ \ \isamarkupfalse%
\isacommand{apply}{\isacharparenleft}simp{\isacharparenright}\isanewline
\ \ \ \ \ \ \ \ \ \ \ \ \ \ \isamarkupfalse%
\isacommand{apply}{\isacharparenleft}subst\ Eot{\isacharunderscore}GetInput{\isacharparenright}\isanewline
\ \ \ \ \ \ \ \ \ \ \ \ \ \ \isamarkupfalse%
\isacommand{by}\ {\isacharparenleft}rule\ ret{\isacharunderscore}one{\isacharunderscore}aux{\isacharparenright}\isanewline
\ \ \ \ \ \ \ \ \ \ \isamarkupfalse%
\isacommand{qed}\isanewline
\ \ \ \ \ \ \ \ \isamarkupfalse%
\isacommand{qed}\isanewline
\ \ \ \ \ \ \ \ \isamarkupfalse%
\isacommand{ultimately}\isanewline
\ \ \ \ \ \ \ \ \isamarkupfalse%
\isacommand{show}\ {\isacharquery}thesis\ \isamarkupfalse%
\isacommand{by}\ {\isacharparenleft}rule\ pdl{\isacharunderscore}plugD{\isacharunderscore}lifted{\isadigit{1}}{\isacharparenright}\isanewline
\ \ \ \ \ \ \isamarkupfalse%
\isacommand{qed}\isanewline
\ \ \ \ \ \ \isamarkupfalse%
\isacommand{thus}\ {\isacharquery}thesis\ \isamarkupfalse%
\isacommand{by}\ {\isacharparenleft}simp\ add{\isacharcolon}\ natp{\isacharunderscore}def\ many{\isadigit{1}}{\isacharunderscore}def\ mon{\isacharunderscore}ctr\ del{\isacharcolon}\ bind{\isacharunderscore}assoc{\isacharparenright}\isanewline
\ \ \ \ \isamarkupfalse%
\isacommand{qed}\isanewline
\ \ \isamarkupfalse%
\isacommand{qed}\isanewline
\ \ \isamarkupfalse%
\isacommand{ultimately}\ \isamarkupfalse%
\isacommand{show}\ {\isacharquery}thesis\ \isamarkupfalse%
\isacommand{by}\ {\isacharparenleft}rule\ pdl{\isacharunderscore}plugD{\isacharparenright}\isanewline
\isamarkupfalse%
\isacommand{qed}\isanewline
\isamarkupfalse%
\isamarkupfalse%
\isamarkuptrue%
\isamarkupfalse%
\isamarkupfalse%
\isamarkupfalse%
\isamarkupfalse%
\isamarkupfalse%
\isamarkupfalse%
\isamarkupfalse%
\isamarkupfalse%
\isamarkupfalse%
\isamarkupfalse%
\isamarkupfalse%
\isamarkupfalse%
\isamarkupfalse%
\isamarkupfalse%
\isamarkupfalse%
\isamarkupfalse%
\isamarkuptrue%
\isamarkupfalse%
\isamarkupfalse%
\isamarkupfalse%
\isamarkupfalse%
\isamarkupfalse%
\isamarkupfalse%
\isamarkupfalse%
\isamarkupfalse%
\isacommand{end}\isanewline
\isamarkupfalse%
\end{isabellebody}%
%%% Local Variables:
%%% mode: latex
%%% TeX-master: "root"
%%% End:
